\documentclass[a4paper,11pt]{article}
\usepackage{pdflscape}
\usepackage[utf8]{inputenc}
\usepackage[T1]{fontenc}
%\usepackage{fourier} % math & rm
%\usepackage{amsthm,amsfonts,amsmath,amssymb,textcomp}
\usepackage{pst-all,pstricks-add,pst-eucl}
\everymath{\displaystyle}
\usepackage{fp,ifthen}
%\usepackage{color}
%\usepackage{graphicx}
\usepackage{setspace}
\usepackage{array}
\usepackage{tabularx}
\usepackage{supertabular}
\usepackage{hhline}
\usepackage{variations}
\usepackage{enumerate}
\usepackage{pifont}
\usepackage{framed}
\usepackage[fleqn]{amsmath}
\usepackage{amssymb}
\usepackage[framed]{ntheorem}
\usepackage{multicol}
\usepackage{kpfonts}
\usepackage{manfnt}

%\usepackage[hmargin=2.5cm, vmargin=2.5cm]{geometry}
\usepackage{vmargin}          % Pour fixer les marges du document
\setmarginsrb
{1.5cm} 	%marge gauche
{0.5cm} 	  %marge en haut
{1.5cm}     %marge droite
{0.5cm}   %marge en bas
{1cm} 	%hauteur de l'entête
{0.5cm}   %distance entre l'entête et le texte
{1cm} 	  %hauteur du pied de page
{0.5cm}     %distance entre le texte et le pied de page

\newcommand{\R}{\mathbb{R}}
\newcommand{\N}{\mathbb{N}}
%\newcommand{\D}{\mathbb{D}}
\newcommand{\Z}{\mathbb{Z}}
\newcommand{\Q}{\mathbb{Q}}
\newcommand{\C}{\mathbb{C}}
\newcommand{\e}{\text{e}}
\newcommand{\dx}{\text{d}x}
\newcommand{\vect}[1]{\mathchoice%
  {\overrightarrow{\displaystyle\mathstrut#1\,\,}}%
  {\overrightarrow{\textstyle\mathstrut#1\,\,}}%
  {\overrightarrow{\scriptstyle\mathstrut#1\,\,}}%
  {\overrightarrow{\scriptscriptstyle\mathstrut#1\,\,}}}
\newcommand\arraybslash{\let\\\@arraycr}
\renewcommand{\theenumi}{\textbf{\arabic{enumi}}}
\renewcommand{\labelenumi}{\textbf{\theenumi.}}
\renewcommand{\theenumii}{\textbf{\alph{enumii}}}
\renewcommand{\labelenumii}{\textbf{\theenumii.}}
\renewcommand{\and}{\wedge}

\theoremstyle{break}
\theorembodyfont{\upshape}
\newframedtheorem{Theo}{Théorème}
\newframedtheorem{Prop}{Propriété}
\newframedtheorem{Def}{Définition}

\newtheorem{Term}{Terminologie}
\newtheorem{Rq}{Remarque}
\newtheorem{Ex}{Exemple}
%\newtheorem{exo}{Exercice}

%\theorembodyfont{\small \sffamily}
%\newtheorem{sol}{solution}

\newenvironment{sol}% 
{\def\FrameCommand{\hspace{0.5cm} {\color{black} \vrule width 1pt} \hspace{-0.7cm}}%
  \framed {\advance\hsize-\width}
  \noindent \small \sffamily  %\underline{Solution :}%\\
}%
{\endframed}

\newrgbcolor{vert}{0 0.4 0}
\newrgbcolor{bistre}{1 .50 .30}
\setlength\tabcolsep{1mm}
\renewcommand\arraystretch{1.3}

\everymath{\displaystyle}
\hyphenpenalty 10000 %supprime toutes les césures
%\setcounter{secnumdepth}{0}
%\newcounter{saveenum}

\usepackage[frenchb]{babel}
\usepackage{fancyhdr,lastpage}
\usepackage{fancybox}

%\headheight 15.0 pt
\fancyhead[L]{Avec des dérivées}
\fancyhead[C]{Contrôle sujet A}
\fancyhead[R]{Mardi 13 décembre 2016}
%\fancyfoot[L]{{\scriptsize\textsl{Cité scolaire de Lorgues}}}
\fancyfoot[C]{}
%\fancyfoot[C]{\scriptsize\thepage}
%\fancyfoot[C]{\scriptsize\thepage/\pageref{LastPage}}

\title{}
\author{}
\date{}

%\pagestyle{empty}
\pagestyle{fancy}
\usepackage[np]{numprint}

\renewcommand\arraystretch{1.8}

\newcounter{numero}
\newcommand{\exo}{
  \addtocounter{numero}{1}%
  \textbf{\underline{Exercice \arabic{numero}:}}\quad}

\frenchbsetup{StandardEnumerateEnv=true}
\usepackage{etex}
\usepackage{pgf,tikz,tkz-tab}
\usepackage{comment}
%\includecomment{correction}
%\renewcommand{correction}{correction}
\excludecomment{correction}
\usepackage{array}


\begin{document}
  \setlength{\unitlength}{1mm}
  \setlength\parindent{0mm}
  
  %\thispagestyle{empty}
  %\exo
  
  \vspace{1cm}
  ~
  
  \begin{exo}(5 points)
  ~
      \vspace{0.25cm}
      
    Soit $f(x)=\frac{x+1}{2x+3}$
    \begin{enumerate}
     \item Donner le plus grand ensemble de réels sur lequel $f$ est
     définie et dérivable.
\begin{correction}

  $]-\infty;-\frac{3}{2}[ \cup ]-\frac{3}{2};+\infty[$ (1pt)
\end{correction}   
     \item Calculer la dérivée de la fonction $f(x)$.
\begin{correction}

  $f'(x)=\frac{(x+1)'(2x+3)-(x+1)(2x+3)'}{(2x+3)^2}$
  $=\frac{(2x+3)-(x+1)(2)}{(2x+3)^2}$
  $=\frac{2x+3-2x-2}{(2x+3)^2}$
  $=\frac{1}{(2x+3)^2}$ (2 pts)
\end{correction} 
     \item \'Etudier le signe de $f'(x)$.
\begin{correction}

  Pour tout r\'eel $x$ diff\'erent de $-\frac{3}{2}$, $f'(x)>0$.
  (1 pt)
\end{correction}

     \item Réaliser le tableau des variations de $f(x)$.
\begin{correction}

    $f$ est croissante sur $]-\infty;-\frac{3}{2}[$. 
    $f$ est aussi croissante sur $]-\frac{3}{2};+\infty[$. (1 pt)
\end{correction}
    \end{enumerate}
  \end{exo}
  
   ~
  \vspace{0.5cm}
  
   \begin{exo}(3 points)
   ~
      \vspace{0.25cm}
   
   Soit $\mathcal{C}$ la courbe représentative de la fonction
   $f$ définie sur $[0;+\infty[$ par $f(x)=\sqrt{x}$.
   
   Soit $T$ la droite d'équation $T:y=\frac{1}{6}x+ \frac{3}{2}$.
   
   $T$ est-elle tangente à la courbe $\mathcal{C}$ ?
\begin{correction}

    $f'(x)=\frac{1}{2\sqrt{x}}$, (1 pt)
    $f'(9)=\frac{1}{2\sqrt{9}}=\frac{1}{6}$, (1 pt)
    $f(9)=3$   
    
    $T_{f,9}:y=\frac{1}{6}(x-9)+3=\frac{1}{6}x+\frac{3}{2}$
    
    $T=T_{f,9}$ est bien tangente \`a $\mathcal{C}$. (1 pt)
\end{correction}
  \end{exo}
    
     ~
  \vspace{0.5cm}
  
    
\begin{exo}(9 points)
~
      \vspace{0.25cm}
      
 Soit $f$ la fonction définie sur $\mathbb{R}$ par:
 \[f(x)=x^3-\frac{9}{2}x^2+6x+5\]
 
 \begin{enumerate}
  \item Calculer $f'(x)$.
\begin{correction}

  $f'(x)=3x^2-9x+6$ (2 pts)
\end{correction}

  \item \'Etudier le signe du trinôme  $3x^2-9x+6$.
\begin{correction}

  $\Delta=81-4(3)(6)=81-72=9=3^2$, $x_1=\frac{9-3}{6}=1$,
  $x_2=\frac{9+3}{6}=2$. La suite dans le tableau de la
  question 3). (3 pts)
\end{correction}

  \item Dresser le tableau de variation de $f$.
\begin{correction}
  
   \begin{tikzpicture}
\tkzTabInit{$x$/1,$f'(x)$/1,$f(x)$/2}{$-\infty$,$1$,
$2$,$+\infty$}
\tkzTabLine{,+,z,-,z,+}
\tkzTabVar{-/,+/,-/,+/}
\end{tikzpicture} (2 pts)

\end{correction}

  \item Déterminer les extremums locaux de $f$.
\begin{correction}

  $f(1)=1^3-\frac{9}{2}1^2+6+5=\frac{15}{2}$
  
  $\frac{15}{2}$ est un maximum local. (1 pt)
  
  $f(2)=2^3-\frac{9}{2}2^2+6(2)+5$=7
  , $7$ est un minimum local pour $f$. (1pt)
\end{correction}  

\begin{item}
 Donner le meilleur encadrement possible pour $f(x)$ lorsque 
 \begin{enumerate}
 \item $x$ appartient à $[2,3]$.
  \item $x$ appartient à $[0,2]$
 \end{enumerate}

\begin{correction}
  $f(3)=-25$ et pour $x \in [2,3]$,  $7 \leq f(x) \leq \frac{19}{2}$.
  
  $f(-3)=-79$ et pour $x \in [0,2]$, $5 \leq f(x) \leq \frac{15}{2}$.
\end{correction} 
 
\end{item}
 \end{enumerate}

\end{exo}

~
      \vspace{0.5cm}
      
\begin{exo}(3 points)
~
      \vspace{0.25cm}
      
  Montrer en utilisant l'expression du taux d'accroissement que la dérivée de la fonction  $f(x)=x^2$ au point d'abscisse $a$ est $f'(a)=2a$.
\end{exo}

\begin{correction}
 $\frac{f(a+h)-f(a)}{h}=\frac{(a+h)^2-a^2}{h}=\frac{a^2-2ah+h^2-a^2}{h}=\frac{2ah+h^2}{h}=2a+h$ qui tend vers 2a quand h tend vers 0.
\end{correction}
~
      \vspace{0.5cm}

\begin{exo}(Bonus)
~
      \vspace{0.25cm}
      
 Montrer par un calcul de dérivée que le sommet d'une parabole d'équation $\mathcal{P}:y=ax^2+bx+c$ a pour abscisse $-\frac{b}{2a}$.
\end{exo}

~
      \vspace{0.5cm}

\begin{exo}(Bonus)
  
 Calculer la dérivée de la fonction définie sur $]-\frac{1}{3},+\infty[$ par $g(x)=\sqrt{3x+1}$.
\end{exo}
  
      ~
  \vspace{0.5cm}
  
\begin{exo}(Bonus)
    ~
  \vspace{0.5cm}
  
    Montrer que pour toute fonction $h$ dérivable sur un intervalle I, la fonction $i(x)=[h(x)]^2$
  est aussi définie et dérivable sur I et $i'(x)=2 h'(x) \times h(x)$.
 
\end{exo}



\end{document}
 ~
  \vspace{1cm}
  

\begin{exo}(3 points)

Soient $ABC$ un triangle non aplati, $K$ le point tel que 
$\vec{AK}=\vec{AB}+2\vec{AC}$, $M$ le milieu de $[AB]$ et
$I$ le milieu de $[MC]$.

   \begin{enumerate}
    \item Calculer les coordonnées des points $I$ et $K$ dans le repère
    $(A;\vec{AB},\vec{AC})$.
    
\begin{correction}

  $K(1;2)$ d'apr\`es l'\'enonc\'e. (1 pt)
  
  $\vec{AI}=\frac{1}{2}\vec{AM}+\frac{1}{2}\vec{AC}
  =\frac{1}{2}(\frac{1}{2}\vec{AB})+\frac{1}{2}\vec{AC}$
  
  $I(\frac{1}{4};\frac{1}{2})$ (1 pt)
\end{correction}

    \item En déduire que les points $A,I$ et $K$ sont alignés.

\begin{correction}

  $\vec{AK}=4\vec{AI}$. $\vec{AK}$ et $\vec{AI}$ sont colin\'eaires
  et les points $A,I,K$ sont donc align\'es. (1 pt)
\end{correction}

   \end{enumerate}

 
\end{exo}


    

