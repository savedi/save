\documentclass[a4paper,11pt]{article}
\usepackage{pdflscape}
\usepackage[utf8]{inputenc}
\usepackage[T1]{fontenc}
%\usepackage{fourier} % math & rm
%\usepackage{amsthm,amsfonts,amsmath,amssymb,textcomp}
\usepackage{pst-all,pstricks-add,pst-eucl}
\everymath{\displaystyle}
\usepackage{fp,ifthen}
%\usepackage{color}
%\usepackage{graphicx}
\usepackage{setspace}
\usepackage{array}
\usepackage{tabularx}
\usepackage{supertabular}
\usepackage{hhline}
\usepackage{variations}
\usepackage{enumerate}
\usepackage{pifont}
\usepackage{framed}
\usepackage[fleqn]{amsmath}
\usepackage{amssymb}
\usepackage[framed]{ntheorem}
\usepackage{multicol}
\usepackage{kpfonts}
\usepackage{manfnt}

%\usepackage[hmargin=2.5cm, vmargin=2.5cm]{geometry}
\usepackage{vmargin}          % Pour fixer les marges du document
\setmarginsrb
{1.5cm} 	%marge gauche
{0.5cm} 	  %marge en haut
{1.5cm}     %marge droite
{0.5cm}   %marge en bas
{1cm} 	%hauteur de l'entête
{0.5cm}   %distance entre l'entête et le texte
{1cm} 	  %hauteur du pied de page
{0.5cm}     %distance entre le texte et le pied de page

\newcommand{\R}{\mathbb{R}}
\newcommand{\N}{\mathbb{N}}
%\newcommand{\D}{\mathbb{D}}
\newcommand{\Z}{\mathbb{Z}}
\newcommand{\Q}{\mathbb{Q}}
\newcommand{\C}{\mathbb{C}}
\newcommand{\e}{\text{e}}
\newcommand{\dx}{\text{d}x}
\newcommand{\vect}[1]{\mathchoice%
  {\overrightarrow{\displaystyle\mathstrut#1\,\,}}%
  {\overrightarrow{\textstyle\mathstrut#1\,\,}}%
  {\overrightarrow{\scriptstyle\mathstrut#1\,\,}}%
  {\overrightarrow{\scriptscriptstyle\mathstrut#1\,\,}}}
\newcommand\arraybslash{\let\\\@arraycr}
\renewcommand{\theenumi}{\textbf{\arabic{enumi}}}
\renewcommand{\labelenumi}{\textbf{\theenumi.}}
\renewcommand{\theenumii}{\textbf{\alph{enumii}}}
\renewcommand{\labelenumii}{\textbf{\theenumii.}}
\renewcommand{\and}{\wedge}

\theoremstyle{break}
\theorembodyfont{\upshape}
\newframedtheorem{Theo}{Théorème}
\newframedtheorem{Prop}{Propriété}
\newframedtheorem{Def}{Définition}

\newtheorem{Term}{Terminologie}
\newtheorem{Rq}{Remarque}
\newtheorem{Ex}{Exemple}
%\newtheorem{exo}{Exercice}

%\theorembodyfont{\small \sffamily}
%\newtheorem{sol}{solution}

\newenvironment{sol}% 
{\def\FrameCommand{\hspace{0.5cm} {\color{black} \vrule width 1pt} \hspace{-0.7cm}}%
  \framed {\advance\hsize-\width}
  \noindent \small \sffamily  %\underline{Solution :}%\\
}%
{\endframed}

\newrgbcolor{vert}{0 0.4 0}
\newrgbcolor{bistre}{1 .50 .30}
\setlength\tabcolsep{1mm}
\renewcommand\arraystretch{1.3}

\everymath{\displaystyle}
\hyphenpenalty 10000 %supprime toutes les césures
%\setcounter{secnumdepth}{0}
%\newcounter{saveenum}

\usepackage[frenchb]{babel}
\usepackage{fancyhdr,lastpage}
\usepackage{fancybox}

%\headheight 15.0 pt
\fancyhead[L]{Avec de la colinéarité}
\fancyhead[C]{Contrôle sujet A}
\fancyhead[R]{Jeudi 17 novembre 2016}
%\fancyfoot[L]{{\scriptsize\textsl{Cité scolaire de Lorgues}}}
\fancyfoot[C]{}
%\fancyfoot[C]{\scriptsize\thepage}
%\fancyfoot[C]{\scriptsize\thepage/\pageref{LastPage}}

\title{}
\author{}
\date{}

%\pagestyle{empty}
\pagestyle{fancy}
\usepackage[np]{numprint}

\renewcommand\arraystretch{1.8}

\newcounter{numero}
\newcommand{\exo}{
  \addtocounter{numero}{1}%
  \textbf{\underline{Exercice \arabic{numero}:}}\quad}

\frenchbsetup{StandardEnumerateEnv=true}
\usepackage{etex}
\usepackage{pgf,tikz,tkz-tab}
\usepackage{comment}
%\includecomment{correction}
%\renewcommand{correction}{correction}
\excludecomment{correction}
\usepackage{array}


\begin{document}
  \setlength{\unitlength}{1mm}
  \setlength\parindent{0mm}
  
  %\thispagestyle{empty}
  %\exo
  
  \vspace{1cm}
  ~
  
  \begin{exo}(8 points)
  ~
      \vspace{0.25cm}
      
      Soit $ABCD$ un parallélogramme non aplati et soient $M$ et $N$ les points tels que :
\begin{center} $\overrightarrow{AM}=\dfrac{2}{3}\overrightarrow{AB}$, \hspace{0.2cm} et $\overrightarrow{CN}=\dfrac{3}{2}\overrightarrow{CB}$ \end{center}
\begin{enumerate}
\item Pourquoi les vecteurs $\vec{AB}$ et $\vec{AD}$ permettent de réaliser un repère $(A;\vec{AB},\vec{AD})$ ?

\begin{correction}
  $ABCD$ \'etant un parall\'elogramme non aplati, les vecteurs $\vec{AB}$ et $\vec{AD}$ ne sont pas colin\'eaires et peuvent ainsi former un rep\`ere $(A;\vec{AB},\vec{AD})$.
  (1pt)
\end{correction}


\item Calculer les coordonnées des points $D$, $M$, $C$ et $N$ dans le repère $(A,\vec{AB},\vec{AD})$.

\begin{correction}
  Comme $\vec{AD}=0 \times \vec{AB}+ 1 \times \vec{AD}$, $D(0;1)$. (1pt)
  
  Comme $\vec{AM}=\dfrac{2}{3}\vec{AB}$, $M(\dfrac{2}{3};0)$. (1pt)
  
  Comme $ABCD$ est un parall\'elogramme $\vec{AC}=1 \times \vec{AB}+1 \vec{AD}$ et $C(1;1)$. (1pt)
  
  Or $\vec{CN}(x_N-1,y_N-1)=\dfrac{3}{2}\vec{CB}(\dfrac{3}{2}(1-1);\dfrac{3}{2}(0-1))=(0;-\dfrac{3}{2})$ d'o\`u $N(1;-\dfrac{1}{2})$. (1pt)
\end{correction}

\item Montrer que les points $D$, $M$ et $N$ sont alignés.

\begin{correction}
$\vec{DM}(\frac{2}{3}-0;0-1)=(\frac{2}{3};-1)$ et $\vec{DN}(1-0;-\frac{1}{2}-1)=(1;-\frac{3}{2})$

$\vec{DN}=\frac{3}{2}\vec{DM}$ donc $\vec{DN}$ est colin\'eaire \`a $\vec{DM}$ et les points $D,N,M$ sont align\'es. (1pt)
\end{correction}

\item On considère un nombre réel $a$ non nul, et $P$ et $Q$ définis par
      \begin{center} $\overrightarrow{AP}=a\overrightarrow{AB}$, \hspace{0.2cm} et $\overrightarrow{CQ}=\dfrac{1}{a}\overrightarrow{CB}$ \end{center}
      Les points $D$, $P$ et $Q$ sont-ils toujours alignés ?
      
\begin{correction}

$P(a;0)$ et $\vec{CQ}(x_Q-1,y_Q-1)=\dfrac{1}{a}\vec{CB}=(\dfrac{1}{a}(1-1);\dfrac{1}{a}(0-1)=(0;-\dfrac{1}{a})$ d'o\`u 
$Q(1;1-\dfrac{1}{a})$; (1pt)

$\vec{DP}(a-0;0-1)=(a;-1)$ et $\vec{DQ}(1-0;1-\dfrac{1}{a}-1)=(1;-\dfrac{1}{a})$.

Ainsi $\vec{DP}$ et $\vec{DQ}$ sont colin\'eaires. Les points $D,P,Q$ sont toujours align\'es. (1pt)
\end{correction}

\end{enumerate}

% \item   
% Mais alors $\vec{CN}$
% \item Montrer que les points $D$, $M$ et $N$ sont alignés.
% \item On considère un nombre réel $a$ non nul, et $P$ et $Q$ définis par
%       \begin{center} $\overrightarrow{AP}=a\overrightarrow{AB}$, \hspace{0.2cm} et $\overrightarrow{CQ}=\dfrac{1}{a}\overrightarrow{CB}$ \end{center}
%       Les points $D$, $P$ et $Q$ sont-ils toujours alignés ?
% 
% \end{correction}

  \end{exo}
  
   ~
  \vspace{0.5cm}
  

   \begin{exo}(11 points)
   Soit $(O;\vec{i},\vec{j})$ un repère du plan.
   \begin{enumerate}
    \item Trouver une équation cartésienne pour la droite $d_1$ qui passe par le point $A(2;5)$ 
    et qui a pour vecteur directeur $\vec{u}(2;1)$. 

\begin{correction}
    
   $d_1$ poss\`ede une \'equation de la forme $d_1:x-2y+c=0$.
   
   $A \in d_1 \Leftrightarrow 2-2 \times 5+c=0 \Leftrightarrow c=8$
   
   $d_1$ admet comme \'equation cart\'esienne $d_1:x-2y+8=0$. (1 pt)

\end{correction}    
    
    \item Donner l'équation réduite de la droite $d_1$.

\begin{correction}
    
 $x-2y+8=0 \Leftrightarrow x+8=2y \Leftrightarrow y=\frac{1}{2}x+4$
   $d_1$ admet comme \'equation r\'eduite $d_1:y=\frac{1}{2}x+4$ (1 pt)

\end{correction}
   
    \item Trouver l'équation réduite de la droite $d_2$ dirigée par le vecteur $\vec{v}(0;2)$ 
    passant par le point $B(4;-3)$.
    
\begin{correction}
 $d_2$ est parall\`ele \`a l'axe des ordonn\'ees, son \'equation r\'eduite est de la forme $d_2:x=k$ et comme $B(4;-3) \in d_2$,
 $d_2:x=4$. (1 pt)
\end{correction}
   
    \item Dans chacun des cas, donner un vecteur directeur:
    
    \begin{itemize}
    \item $u_3$ de la droite $d_3:3x-5y+1=0$.
    \item $u_4$ de la droite $d_4:y=3x-7$.
    \item $u_5$ de la droite $d_5:x-6=0$.
     
    \end{itemize}

    
\begin{correction}
 $\vec{u_3}(5;3)$ est un vecteur directeur directeur de $d_3$. (1 pt)
 
 $\vec{u}_4(1;3)$ est un vecteur directeur de $d_4$. (1pt) 
 
 $\vec{u}_5(0;1)$ est un vecteur directeur de $d_5$. (1pt)
\end{correction}
    
\item Donner l'ordonnée à l'origine de la droite $d_6:7x+5y-15=0$.
    
\begin{correction}
  Le point de coordonn\'ees $(0;3)$ est sur $d_6$ et $3$ est donc l'ordonn\'ee \`a 
  l'origine de $d_6$. (1 pt)
\end{correction}

  \item Déterminer si le point $E(-1;2)$ appartient à la droite 
  $d_7= \frac{x}{5}+\frac{y}{9}=0$. 
\begin{correction}
 $\frac{-1}{5}+\frac{2}{9}=-\frac{9}{45}+\frac{10}{45} = \frac{1}{45} \neq 0$ et $E \notin d_7$. (1 pt)
\end{correction}

    \item Trouver toutes les valeurs de $m$ pour lesquelles $d_8:3x+my+4=0$ est parall\`ele 
    à la droite $d_9:5x-4y+1=0$

\begin{correction}
 $d_8$ est parall\`ele \`a $d_9$ 
 
 $\Leftrightarrow$
 
 les vecteurs directeurs $\vec{u}(-m;3)$ et $\vec{v}(4;5)$ sont colin\'eaires. (1 pt)
 
 $\Leftrightarrow$
 
 $-5m-12=0$ soit $m=-\frac{12}{5}$. $-\frac{12}{5}$ est la seule valeur de $m$ pour laquelle 
 les droites $d_8$ et $d_9$ sont parall\`eles. (1 pt)
\end{correction}

\item Trouver une équation cartésienne pour la droite $(AB)$ avec $A(3;7)$ et $B(2;5)$.

\begin{correction}
 $\vec{AB}(2-3;5-7)=(-1;-2)$ est un vecteur directeur de $(AB)$, $\vec{u}(1;2)$ aussi. La droite $(AB)$ poss\`ede une \'equation
 de la forme $2x-y+c=0$. Or $B(2;5)$ appartient \`a $(AB)$ d'o\`u $2 \times 2-5+c=0$ et $c=1$. 
 
 En d\'efinitive, $(AB):2x-y+1=0$. (1pt)
\end{correction}


    \end{enumerate}
  \end{exo}
    
     ~
  \vspace{0.5cm}
  
  \begin{exo}(1 point)
   
  
  Résoudre l'inéquation $\frac{2}{3x+1} > 2x+1$.
  
\begin{correction}

 $\frac{2}{3x+1} > 2x+1$
 
 $\Leftrightarrow$
 
 $\frac{2}{3x+1} - (2x+1) >0$
 
 $\Leftrightarrow$
 
 $\frac{2-(3x+1)(2x+1)}{3x+1}>0$
 
 $\Leftrightarrow$
 
 $\frac{-6x^2-5x+1}{3x+1}>0$ (0.5 pt)
 
 $\Delta=25+24=7^2$, $x_1=\frac{-(-5)-7}{2\times (-6)}=\frac{1}{6}$ et $x_2=\frac{-(-5)+7}{2\times (-6)}=-1$. 
 
 On pose $f(x)=-6x^2-5x+1$ et $g(x)=\frac{-6x^2-5x+1}{3x+1}$
 
 \begin{tikzpicture}
\tkzTabInit{$x$/1,$f(x)$/1,$3x+1$/1,$g(x)$/1}{$-\infty$,$-1$,$-\frac{1}{3}$,$\frac{1}{6}$,$+\infty$}
\tkzTabLine{,-,z,,+,,z,-}
\tkzTabLine{,,-,,z,,+,}
\tkzTabLine{,+,z,-,d,+,z,-}
\end{tikzpicture}
 
 d'o\`u $S=\left ]-\infty;-1\right [ \cup \left ]-\frac{1}{3};\frac{1}{6} \right [$ (0.5 pt)
\end{correction}
\end{exo}
    
\end{document}
 


\begin{exo}(2 points)
   \begin{enumerate}
  \item Montrer que les droites $d_9:-3 x - y +6=0$ et $d_{10}:x-2 y +5=0$ sont sécantes et calculer 
  les coordonnées de leur point d'intersection.

\begin{correction}
 Les vecteurs $\vec{u}(1;-3)$ et $\vec{v}(2;1)$ dirigent $d_9$ et $d_{10}$ et ne sont pas colin\'eaires
 ($(1)(1)-(-3)(2) \neq 0$). (1 pt)
 
  Un point $M(x;y)$ est sur $d_9$ et $d_{10}$
  
  $\Leftrightarrow$
  
  $y=-3x+6$ et $x-2(-3x+6)+5=0$ 
  
  $\Leftrightarrow$  
  
  $x=1$ et $y=3$
  
  Le point d'intersection de $d_9$ et $d_{10}$ est le point $P(1;3)$. (1 pt)
\end{correction}
  
   \end{enumerate}

 
\end{exo}

$\sqrt{2}x^2-(2+\sqrt{2})x+2=0$.