\documentclass[a4paper,11pt]{article}
\usepackage{pdflscape}
\usepackage[utf8]{inputenc}
\usepackage[T1]{fontenc}
%\usepackage{fourier} % math & rm
%\usepackage{amsthm,amsfonts,amsmath,amssymb,textcomp}
\usepackage{pst-all,pstricks-add,pst-eucl}
\everymath{\displaystyle}
\usepackage{fp,ifthen}
%\usepackage{color}
%\usepackage{graphicx}
\usepackage{setspace}
\usepackage{array}
\usepackage{tabularx}
\usepackage{supertabular}
\usepackage{hhline}
\usepackage{variations}
\usepackage{enumerate}
\usepackage{pifont}
\usepackage{framed}
\usepackage[fleqn]{amsmath}
\usepackage{amssymb}
\usepackage[framed]{ntheorem}
\usepackage{multicol}
\usepackage{kpfonts}
\usepackage{manfnt}

%\usepackage[hmargin=2.5cm, vmargin=2.5cm]{geometry}
\usepackage{vmargin}          % Pour fixer les marges du document
\setmarginsrb
{1.5cm} 	%marge gauche
{0.5cm} 	  %marge en haut
{1.5cm}     %marge droite
{0.5cm}   %marge en bas
{1cm} 	%hauteur de l'entête
{0.5cm}   %distance entre l'entête et le texte
{1cm} 	  %hauteur du pied de page
{0.5cm}     %distance entre le texte et le pied de page

\newcommand{\R}{\mathbb{R}}
\newcommand{\N}{\mathbb{N}}
%\newcommand{\D}{\mathbb{D}}
\newcommand{\Z}{\mathbb{Z}}
\newcommand{\Q}{\mathbb{Q}}
\newcommand{\C}{\mathbb{C}}
\newcommand{\e}{\text{e}}
\newcommand{\dx}{\text{d}x}
\newcommand{\vect}[1]{\mathchoice%
  {\overrightarrow{\displaystyle\mathstrut#1\,\,}}%
  {\overrightarrow{\textstyle\mathstrut#1\,\,}}%
  {\overrightarrow{\scriptstyle\mathstrut#1\,\,}}%
  {\overrightarrow{\scriptscriptstyle\mathstrut#1\,\,}}}
\newcommand\arraybslash{\let\\\@arraycr}
\renewcommand{\theenumi}{\textbf{\arabic{enumi}}}
\renewcommand{\labelenumi}{\textbf{\theenumi.}}
\renewcommand{\theenumii}{\textbf{\alph{enumii}}}
\renewcommand{\labelenumii}{\textbf{\theenumii.}}
\renewcommand{\and}{\wedge}

\theoremstyle{break}
\theorembodyfont{\upshape}
\newframedtheorem{Theo}{Théorème}
\newframedtheorem{Prop}{Propriété}
\newframedtheorem{Def}{Définition}

\newtheorem{Term}{Terminologie}
\newtheorem{Rq}{Remarque}
\newtheorem{Ex}{Exemple}
%\newtheorem{exo}{Exercice}

%\theorembodyfont{\small \sffamily}
%\newtheorem{sol}{solution}

\newenvironment{sol}% 
{\def\FrameCommand{\hspace{0.5cm} {\color{black} \vrule width 1pt} \hspace{-0.7cm}}%
  \framed {\advance\hsize-\width}
  \noindent \small \sffamily  %\underline{Solution :}%\\
}%
{\endframed}

\newrgbcolor{vert}{0 0.4 0}
\newrgbcolor{bistre}{1 .50 .30}
\setlength\tabcolsep{1mm}
\renewcommand\arraystretch{1.3}

\everymath{\displaystyle}
\hyphenpenalty 10000 %supprime toutes les césures
%\setcounter{secnumdepth}{0}
%\newcounter{saveenum}

\usepackage[frenchb]{babel}
\usepackage{fancyhdr,lastpage}
\usepackage{fancybox}

%\headheight 15.0 pt
\fancyhead[L]{Avec des statistiques et des probabilités}
\fancyhead[C]{Contrôle sujet A}
\fancyhead[R]{Jeudi 30 mars 2017}
%\fancyfoot[L]{{\scriptsize\textsl{Cité scolaire de Lorgues}}}
\fancyfoot[C]{}
%\fancyfoot[C]{\scriptsize\thepage}
%\fancyfoot[C]{\scriptsize\thepage/\pageref{LastPage}}

\title{}
\author{}
\date{}

%\pagestyle{empty}
\pagestyle{fancy}
\usepackage[np]{numprint}

\renewcommand\arraystretch{1.8}

\newcounter{numero}
\newcommand{\exo}{
  \addtocounter{numero}{1}%
  \textbf{\underline{Exercice \arabic{numero}:}}\quad}

\frenchbsetup{StandardEnumerateEnv=true}
\usepackage{etex}
\usepackage{pgf,tikz,tkz-tab}
\usepackage{comment}
\includecomment{correction}
%\renewcommand{correction}{correction}
%\excludecomment{correction}
\usepackage{array}


\begin{document}
  \setlength{\unitlength}{1mm}
  \setlength\parindent{0mm}
  
  %\thispagestyle{empty}
  %\exo
  
 
  \begin{exo}(6 points)
  ~
      \vspace{0.25cm}
      
 
 
 Un sac contient 150 jetons dont 90 sont gris et les autres sont marron.\\
 On tire au hasard, successivement et avec remise 10 jetons de ce sac.
 
  \begin{enumerate}
  \item Identifier et justifier la loi que suit la variable aléatoire comptant le nombre de jetons 
  gris obtenus.
  \item Calculer la probabilité  d'obtenir au moins deux jetons gris (à $10^{-4}$ près).
  \item Si on répète un grand nombre de fois cette expérience, combien de jetons gris obtient-on en moyenne ?
 \end{enumerate}

\begin{correction}

Correction
\begin{enumerate}
 \item Comme on r\'ep\`ete 10 fois de mani\`ere ind\'ependante la m\^eme \'epreuve de Bernoulli 
 (succ\`es:<<obtenir un jeton gris>>) de param\`etre $p=\frac{90}{150}=\frac{6}{10}$, on a un 
 sch\'ema de Bernoulli d'ordre $n=10$.\\
 La variable al\'eatoire $X$ comptant le nombre de jetons gris obtenus suit la loi binomiale $B(10;0.6)$.
 \item On cherche $P(X \geq 2)$.\\
 $P(X \geq 2)=1-P(X=0)-P(X=1)=
 1-C_{10}^0 \times (0.6)^0 \times (0.4)^{10}-C_{10}^1 \times (0.6)^1 \times (0.4)^9 \simeq 0.9983$.
 \item $E(X)=10\times 0.6=6$. Si on r\'eit\`ere un grand nombre de fois cette exp\'erience, on obtient 
 6 jetons gris en moyenne.
\end{enumerate}
  
\end{correction}

  \end{exo}
         ~
  
   \begin{exo}(5 points)
   
      \begin{tabular}{c|c c}
  \begin{minipage}[b]{0.47\textwidth}
  
  Algorithme 1 \\

   $N$ prend la valeur 0\\
   
   $R$ prend la valeur ...\\
   
   Tant que $R \leq $...\\
   
   $N$ prend la valeur ...\\
   
   $R$ prend la valeur ...\\ 
   
   Fin Tant que \\
   
   Afficher $R$

  \end{minipage}
  &
&
  \begin{minipage}[b]{0.47\textwidth}
   
   Algorithme 2\\
 
 $S$ prend la valeur 0\\
 
 Pour $I$ allant de 1 à 8\\
 
 $S$ prend la valeur $S+I^2$ $(*)$\\
 
 Fin Pour\\
 
 Afficher $S$\\
 
\vspace{1.5cm}


  \end{minipage}

  \end{tabular}
  
  \begin{enumerate}
   \item L'algorithme incomplet 1 permet de calculer puis afficher
   la plus petite puissance de 2 dépassant 10000.
   Recopier-le sur votre feuille en le complétant.
   
      \item Donner les valeurs successives des variables $I$ et $S$ à la fin de 
   l'exécution de l'instruction $(*)$ lors des quatre premières itérations de la boucle de 
   l'algorithme 2.
 
 \item Que permet de calculer puis afficher l'algorithme 2 ?
 
  \end{enumerate}

\begin{correction}

Correction
\begin{enumerate}
 \item 
 
 $N$ prend la valeur 0\\
   
   $R$ prend la valeur 1 (0.5 pt)\\
   
   Tant que $R \leq 10000$ (0.5 pt)\\
   
   $N$ prend la valeur $N+1$ (0.5 pt)\\
   
   $R$ prend la valeur $2 \times R$ (0.5 pt)\\ 
   
   Fin Tant que \\
   
   Afficher $R$
 
 
 \item $I=1,S=1$ $I=2$ $S=1+4=5$; $I=3$ $S=5+9=14$; $I=4$ $S=14+16=30$. (2 pts)
  
 \item L'algorithme permet de calculer la somme des carr\'es des 8 premiers entiers. (1 pt)
 
\end{enumerate}

\end{correction}

  \end{exo}
 ~
    
\begin{exo}(4 points)
      
 Soit $f(x)=x^2+x-1$     
\begin{enumerate}
 \item Calculer le discriminant et les racines du trinôme $f(x)$.
 \item Réaliser le tabeau de signe de $f(x)$.
 \item Résoudre l'inéquation $x^2+2 x+1<2+x$ (vous pouvez utiliser l'étude précédente).
\end{enumerate}



\begin{correction}
Correction
\begin{enumerate}
 \item $\Delta=1+4=5$ (0.5 pt), $x_1=\frac{-1-\sqrt{5}}{2}$, $x_2=\frac{-1+\sqrt{5}}{2}$ (1 pt).
 \item $f(x)$ est n\'egatif sur l'intervalle $[x_1;x_2]$ positif en dehors (1 pt).
 \item $x^2+2 x+1<2+x \Leftrightarrow x^2+x-1 <0$ (0.5 pt) , $S=]x_1;x_2[$ (1 pt).

\end{enumerate}

\end{correction}

\end{exo}
~
   
\begin{exo}(5 points)
      
Lors d'une saison footballistique comprenant 38 matchs, deux équipes
ont encaissé des buts selon les répartitions suivantes:

\begin{center}

\begin{tabular}{l l}
 
 Equipe A: 

\begin{tabular}{|l|c|c|c|c|c|}
  \hline
  Nombres de buts  & 0  & 1 &3 & 4 & 5 \\
  \hline
  Nombre de matchs & 23 & 7 &5 & 1 &2 \\
  
  \hline
\end{tabular}

 &

 Equipe B:

\begin{tabular}{|l|c|c|c|c|}
  \hline
  Nombres de buts  & 0  & 1 &3 & 6 \\
  \hline
  Nombre de matchs & 19 & 12 &6 & 1  \\
  
  \hline
\end{tabular}

 \end{tabular}

\end{center} 
\vspace{0.5cm}
Soit $(a_i,n_i)_{1 \leq i \leq 5}$ (resp $(b_j,m_j)_{1 \leq i \leq 4}$) la s\'erie statistique
associ\'ee \'a l'\'equipe $1$ (resp. 2).
\begin{enumerate}
 \item Ecrire une formule littérale pour calculer le nombre moyen de buts
  encaissés par match pour chacune de ces deux équipes (Bonus utiliser un symbole de somme).
  \item Donner une valeur exacte (à l'aide d'une fraction) pour ces deux moyennes.
  \item Ecrire une formule littérale pour calculer l'écart-type de la série
  des nombres de buts encaissés par match pour chacune des deux équipes 
  (Bonus utiliser un symbole de somme).
  \item Donner une valeur approchées au dixième de but près pour ces deux écart-types.
  \item Au vu de ces résultats, quelle équipe possède des performances défensives les plus 
  irrégulières ?
\end{enumerate}

\begin{correction}

Correction 

\begin{enumerate}
 \item  $\overline{a}=\sum_{i=1}^5 \frac{n_i a_i}{38}=\frac{36}{38}=\frac{18}{19}$. (1 pt)
 et $\overline{b}=\sum_{i=1}^4 \frac{m_i b_i}{38}=\frac{36}{38}=\frac{18}{19}$. (1 pt)
 \item $\sigma_{a}=\sqrt{\sum_{i=1}^5 \frac{n_i (a_i-\overline{a})^2}{38}} \simeq 1.5$ (1 pt)
 et $\sigma_{b}=\sqrt{\sum_{i=1}^4 \frac{m_i (b_i-\overline{b})^2}{38}}\simeq 1.3$. (1 pt)
  \item L'\'equipe B poss\`ede les performances d\'efensives les plus r\' eguli\`eres sur cette saison.
\end{enumerate}

\end{correction}

\end{exo}
    
\end{document}
